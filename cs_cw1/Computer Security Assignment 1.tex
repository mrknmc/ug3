
% Default to the notebook output style

    


% Inherit from the specified cell style.




    
\documentclass{article}

    
    
    \usepackage{graphicx} % Used to insert images
    \usepackage{adjustbox} % Used to constrain images to a maximum size 
    \usepackage{color} % Allow colors to be defined
    \usepackage{enumerate} % Needed for markdown enumerations to work
    \usepackage{geometry} % Used to adjust the document margins
    \usepackage{amsmath} % Equations
    \usepackage{amssymb} % Equations
    \usepackage[mathletters]{ucs} % Extended unicode (utf-8) support
    \usepackage[utf8x]{inputenc} % Allow utf-8 characters in the tex document
    \usepackage{fancyvrb} % verbatim replacement that allows latex
    \usepackage{grffile} % extends the file name processing of package graphics 
                         % to support a larger range 
    % The hyperref package gives us a pdf with properly built
    % internal navigation ('pdf bookmarks' for the table of contents,
    % internal cross-reference links, web links for URLs, etc.)
    \usepackage{hyperref}
    \usepackage{longtable} % longtable support required by pandoc >1.10
    

    
    
    \definecolor{orange}{cmyk}{0,0.4,0.8,0.2}
    \definecolor{darkorange}{rgb}{.71,0.21,0.01}
    \definecolor{darkgreen}{rgb}{.12,.54,.11}
    \definecolor{myteal}{rgb}{.26, .44, .56}
    \definecolor{gray}{gray}{0.45}
    \definecolor{lightgray}{gray}{.95}
    \definecolor{mediumgray}{gray}{.8}
    \definecolor{inputbackground}{rgb}{.95, .95, .85}
    \definecolor{outputbackground}{rgb}{.95, .95, .95}
    \definecolor{traceback}{rgb}{1, .95, .95}
    % ansi colors
    \definecolor{red}{rgb}{.6,0,0}
    \definecolor{green}{rgb}{0,.65,0}
    \definecolor{brown}{rgb}{0.6,0.6,0}
    \definecolor{blue}{rgb}{0,.145,.698}
    \definecolor{purple}{rgb}{.698,.145,.698}
    \definecolor{cyan}{rgb}{0,.698,.698}
    \definecolor{lightgray}{gray}{0.5}
    
    % bright ansi colors
    \definecolor{darkgray}{gray}{0.25}
    \definecolor{lightred}{rgb}{1.0,0.39,0.28}
    \definecolor{lightgreen}{rgb}{0.48,0.99,0.0}
    \definecolor{lightblue}{rgb}{0.53,0.81,0.92}
    \definecolor{lightpurple}{rgb}{0.87,0.63,0.87}
    \definecolor{lightcyan}{rgb}{0.5,1.0,0.83}
    
    % commands and environments needed by pandoc snippets
    % extracted from the output of `pandoc -s`
    
    \DefineShortVerb[commandchars=\\\{\}]{\|}
    \DefineVerbatimEnvironment{Highlighting}{Verbatim}{commandchars=\\\{\}}
    % Add ',fontsize=\small' for more characters per line
    \newenvironment{Shaded}{}{}
    \newcommand{\KeywordTok}[1]{\textcolor[rgb]{0.00,0.44,0.13}{\textbf{{#1}}}}
    \newcommand{\DataTypeTok}[1]{\textcolor[rgb]{0.56,0.13,0.00}{{#1}}}
    \newcommand{\DecValTok}[1]{\textcolor[rgb]{0.25,0.63,0.44}{{#1}}}
    \newcommand{\BaseNTok}[1]{\textcolor[rgb]{0.25,0.63,0.44}{{#1}}}
    \newcommand{\FloatTok}[1]{\textcolor[rgb]{0.25,0.63,0.44}{{#1}}}
    \newcommand{\CharTok}[1]{\textcolor[rgb]{0.25,0.44,0.63}{{#1}}}
    \newcommand{\StringTok}[1]{\textcolor[rgb]{0.25,0.44,0.63}{{#1}}}
    \newcommand{\CommentTok}[1]{\textcolor[rgb]{0.38,0.63,0.69}{\textit{{#1}}}}
    \newcommand{\OtherTok}[1]{\textcolor[rgb]{0.00,0.44,0.13}{{#1}}}
    \newcommand{\AlertTok}[1]{\textcolor[rgb]{1.00,0.00,0.00}{\textbf{{#1}}}}
    \newcommand{\FunctionTok}[1]{\textcolor[rgb]{0.02,0.16,0.49}{{#1}}}
    \newcommand{\RegionMarkerTok}[1]{{#1}}
    \newcommand{\ErrorTok}[1]{\textcolor[rgb]{1.00,0.00,0.00}{\textbf{{#1}}}}
    \newcommand{\NormalTok}[1]{{#1}}
    
    % Define a nice break command that doesn't care if a line doesn't already
    % exist.
    \def\br{\hspace*{\fill} \\* }
    % Math Jax compatability definitions
    \def\gt{>}
    \def\lt{<}
    % Document parameters
    \title{Computer Security Assignment 1}
    
    
    

    % Pygments definitions
    
\makeatletter
\def\PY@reset{\let\PY@it=\relax \let\PY@bf=\relax%
    \let\PY@ul=\relax \let\PY@tc=\relax%
    \let\PY@bc=\relax \let\PY@ff=\relax}
\def\PY@tok#1{\csname PY@tok@#1\endcsname}
\def\PY@toks#1+{\ifx\relax#1\empty\else%
    \PY@tok{#1}\expandafter\PY@toks\fi}
\def\PY@do#1{\PY@bc{\PY@tc{\PY@ul{%
    \PY@it{\PY@bf{\PY@ff{#1}}}}}}}
\def\PY#1#2{\PY@reset\PY@toks#1+\relax+\PY@do{#2}}

\expandafter\def\csname PY@tok@gd\endcsname{\def\PY@tc##1{\textcolor[rgb]{0.63,0.00,0.00}{##1}}}
\expandafter\def\csname PY@tok@gu\endcsname{\let\PY@bf=\textbf\def\PY@tc##1{\textcolor[rgb]{0.50,0.00,0.50}{##1}}}
\expandafter\def\csname PY@tok@gt\endcsname{\def\PY@tc##1{\textcolor[rgb]{0.00,0.27,0.87}{##1}}}
\expandafter\def\csname PY@tok@gs\endcsname{\let\PY@bf=\textbf}
\expandafter\def\csname PY@tok@gr\endcsname{\def\PY@tc##1{\textcolor[rgb]{1.00,0.00,0.00}{##1}}}
\expandafter\def\csname PY@tok@cm\endcsname{\let\PY@it=\textit\def\PY@tc##1{\textcolor[rgb]{0.25,0.50,0.50}{##1}}}
\expandafter\def\csname PY@tok@vg\endcsname{\def\PY@tc##1{\textcolor[rgb]{0.10,0.09,0.49}{##1}}}
\expandafter\def\csname PY@tok@m\endcsname{\def\PY@tc##1{\textcolor[rgb]{0.40,0.40,0.40}{##1}}}
\expandafter\def\csname PY@tok@mh\endcsname{\def\PY@tc##1{\textcolor[rgb]{0.40,0.40,0.40}{##1}}}
\expandafter\def\csname PY@tok@go\endcsname{\def\PY@tc##1{\textcolor[rgb]{0.53,0.53,0.53}{##1}}}
\expandafter\def\csname PY@tok@ge\endcsname{\let\PY@it=\textit}
\expandafter\def\csname PY@tok@vc\endcsname{\def\PY@tc##1{\textcolor[rgb]{0.10,0.09,0.49}{##1}}}
\expandafter\def\csname PY@tok@il\endcsname{\def\PY@tc##1{\textcolor[rgb]{0.40,0.40,0.40}{##1}}}
\expandafter\def\csname PY@tok@cs\endcsname{\let\PY@it=\textit\def\PY@tc##1{\textcolor[rgb]{0.25,0.50,0.50}{##1}}}
\expandafter\def\csname PY@tok@cp\endcsname{\def\PY@tc##1{\textcolor[rgb]{0.74,0.48,0.00}{##1}}}
\expandafter\def\csname PY@tok@gi\endcsname{\def\PY@tc##1{\textcolor[rgb]{0.00,0.63,0.00}{##1}}}
\expandafter\def\csname PY@tok@gh\endcsname{\let\PY@bf=\textbf\def\PY@tc##1{\textcolor[rgb]{0.00,0.00,0.50}{##1}}}
\expandafter\def\csname PY@tok@ni\endcsname{\let\PY@bf=\textbf\def\PY@tc##1{\textcolor[rgb]{0.60,0.60,0.60}{##1}}}
\expandafter\def\csname PY@tok@nl\endcsname{\def\PY@tc##1{\textcolor[rgb]{0.63,0.63,0.00}{##1}}}
\expandafter\def\csname PY@tok@nn\endcsname{\let\PY@bf=\textbf\def\PY@tc##1{\textcolor[rgb]{0.00,0.00,1.00}{##1}}}
\expandafter\def\csname PY@tok@no\endcsname{\def\PY@tc##1{\textcolor[rgb]{0.53,0.00,0.00}{##1}}}
\expandafter\def\csname PY@tok@na\endcsname{\def\PY@tc##1{\textcolor[rgb]{0.49,0.56,0.16}{##1}}}
\expandafter\def\csname PY@tok@nb\endcsname{\def\PY@tc##1{\textcolor[rgb]{0.00,0.50,0.00}{##1}}}
\expandafter\def\csname PY@tok@nc\endcsname{\let\PY@bf=\textbf\def\PY@tc##1{\textcolor[rgb]{0.00,0.00,1.00}{##1}}}
\expandafter\def\csname PY@tok@nd\endcsname{\def\PY@tc##1{\textcolor[rgb]{0.67,0.13,1.00}{##1}}}
\expandafter\def\csname PY@tok@ne\endcsname{\let\PY@bf=\textbf\def\PY@tc##1{\textcolor[rgb]{0.82,0.25,0.23}{##1}}}
\expandafter\def\csname PY@tok@nf\endcsname{\def\PY@tc##1{\textcolor[rgb]{0.00,0.00,1.00}{##1}}}
\expandafter\def\csname PY@tok@si\endcsname{\let\PY@bf=\textbf\def\PY@tc##1{\textcolor[rgb]{0.73,0.40,0.53}{##1}}}
\expandafter\def\csname PY@tok@s2\endcsname{\def\PY@tc##1{\textcolor[rgb]{0.73,0.13,0.13}{##1}}}
\expandafter\def\csname PY@tok@vi\endcsname{\def\PY@tc##1{\textcolor[rgb]{0.10,0.09,0.49}{##1}}}
\expandafter\def\csname PY@tok@nt\endcsname{\let\PY@bf=\textbf\def\PY@tc##1{\textcolor[rgb]{0.00,0.50,0.00}{##1}}}
\expandafter\def\csname PY@tok@nv\endcsname{\def\PY@tc##1{\textcolor[rgb]{0.10,0.09,0.49}{##1}}}
\expandafter\def\csname PY@tok@s1\endcsname{\def\PY@tc##1{\textcolor[rgb]{0.73,0.13,0.13}{##1}}}
\expandafter\def\csname PY@tok@sh\endcsname{\def\PY@tc##1{\textcolor[rgb]{0.73,0.13,0.13}{##1}}}
\expandafter\def\csname PY@tok@sc\endcsname{\def\PY@tc##1{\textcolor[rgb]{0.73,0.13,0.13}{##1}}}
\expandafter\def\csname PY@tok@sx\endcsname{\def\PY@tc##1{\textcolor[rgb]{0.00,0.50,0.00}{##1}}}
\expandafter\def\csname PY@tok@bp\endcsname{\def\PY@tc##1{\textcolor[rgb]{0.00,0.50,0.00}{##1}}}
\expandafter\def\csname PY@tok@c1\endcsname{\let\PY@it=\textit\def\PY@tc##1{\textcolor[rgb]{0.25,0.50,0.50}{##1}}}
\expandafter\def\csname PY@tok@kc\endcsname{\let\PY@bf=\textbf\def\PY@tc##1{\textcolor[rgb]{0.00,0.50,0.00}{##1}}}
\expandafter\def\csname PY@tok@c\endcsname{\let\PY@it=\textit\def\PY@tc##1{\textcolor[rgb]{0.25,0.50,0.50}{##1}}}
\expandafter\def\csname PY@tok@mf\endcsname{\def\PY@tc##1{\textcolor[rgb]{0.40,0.40,0.40}{##1}}}
\expandafter\def\csname PY@tok@err\endcsname{\def\PY@bc##1{\setlength{\fboxsep}{0pt}\fcolorbox[rgb]{1.00,0.00,0.00}{1,1,1}{\strut ##1}}}
\expandafter\def\csname PY@tok@kd\endcsname{\let\PY@bf=\textbf\def\PY@tc##1{\textcolor[rgb]{0.00,0.50,0.00}{##1}}}
\expandafter\def\csname PY@tok@ss\endcsname{\def\PY@tc##1{\textcolor[rgb]{0.10,0.09,0.49}{##1}}}
\expandafter\def\csname PY@tok@sr\endcsname{\def\PY@tc##1{\textcolor[rgb]{0.73,0.40,0.53}{##1}}}
\expandafter\def\csname PY@tok@mo\endcsname{\def\PY@tc##1{\textcolor[rgb]{0.40,0.40,0.40}{##1}}}
\expandafter\def\csname PY@tok@kn\endcsname{\let\PY@bf=\textbf\def\PY@tc##1{\textcolor[rgb]{0.00,0.50,0.00}{##1}}}
\expandafter\def\csname PY@tok@mi\endcsname{\def\PY@tc##1{\textcolor[rgb]{0.40,0.40,0.40}{##1}}}
\expandafter\def\csname PY@tok@gp\endcsname{\let\PY@bf=\textbf\def\PY@tc##1{\textcolor[rgb]{0.00,0.00,0.50}{##1}}}
\expandafter\def\csname PY@tok@o\endcsname{\def\PY@tc##1{\textcolor[rgb]{0.40,0.40,0.40}{##1}}}
\expandafter\def\csname PY@tok@kr\endcsname{\let\PY@bf=\textbf\def\PY@tc##1{\textcolor[rgb]{0.00,0.50,0.00}{##1}}}
\expandafter\def\csname PY@tok@s\endcsname{\def\PY@tc##1{\textcolor[rgb]{0.73,0.13,0.13}{##1}}}
\expandafter\def\csname PY@tok@kp\endcsname{\def\PY@tc##1{\textcolor[rgb]{0.00,0.50,0.00}{##1}}}
\expandafter\def\csname PY@tok@w\endcsname{\def\PY@tc##1{\textcolor[rgb]{0.73,0.73,0.73}{##1}}}
\expandafter\def\csname PY@tok@kt\endcsname{\def\PY@tc##1{\textcolor[rgb]{0.69,0.00,0.25}{##1}}}
\expandafter\def\csname PY@tok@ow\endcsname{\let\PY@bf=\textbf\def\PY@tc##1{\textcolor[rgb]{0.67,0.13,1.00}{##1}}}
\expandafter\def\csname PY@tok@sb\endcsname{\def\PY@tc##1{\textcolor[rgb]{0.73,0.13,0.13}{##1}}}
\expandafter\def\csname PY@tok@k\endcsname{\let\PY@bf=\textbf\def\PY@tc##1{\textcolor[rgb]{0.00,0.50,0.00}{##1}}}
\expandafter\def\csname PY@tok@se\endcsname{\let\PY@bf=\textbf\def\PY@tc##1{\textcolor[rgb]{0.73,0.40,0.13}{##1}}}
\expandafter\def\csname PY@tok@sd\endcsname{\let\PY@it=\textit\def\PY@tc##1{\textcolor[rgb]{0.73,0.13,0.13}{##1}}}

\def\PYZbs{\char`\\}
\def\PYZus{\char`\_}
\def\PYZob{\char`\{}
\def\PYZcb{\char`\}}
\def\PYZca{\char`\^}
\def\PYZam{\char`\&}
\def\PYZlt{\char`\<}
\def\PYZgt{\char`\>}
\def\PYZsh{\char`\#}
\def\PYZpc{\char`\%}
\def\PYZdl{\char`\$}
\def\PYZhy{\char`\-}
\def\PYZsq{\char`\'}
\def\PYZdq{\char`\"}
\def\PYZti{\char`\~}
% for compatibility with earlier versions
\def\PYZat{@}
\def\PYZlb{[}
\def\PYZrb{]}
\makeatother


    % Exact colors from NB
    \definecolor{incolor}{rgb}{0.0, 0.0, 0.5}
    \definecolor{outcolor}{rgb}{0.545, 0.0, 0.0}



    
    % Prevent overflowing lines due to hard-to-break entities
    \sloppy 
    % Setup hyperref package
    \hypersetup{
      breaklinks=true,  % so long urls are correctly broken across lines
      colorlinks=true,
      urlcolor=blue,
      linkcolor=darkorange,
      citecolor=darkgreen,
      }
    % Slightly bigger margins than the latex defaults
    
    \geometry{verbose,tmargin=1in,bmargin=1in,lmargin=1in,rmargin=1in}
    
    

    \begin{document}
    
    
    \maketitle
    
    

    
    \section{Computer Security Assignment
1}\label{computer-security-assignment-1}

\subsection{Exercise 1}\label{exercise-1}

\subsubsection{a)}\label{a}

We use the Python library gmpy to verify the requirements for the
ElGamal algorithm.

    \begin{Verbatim}[commandchars=\\\{\}]
{\color{incolor}In [{\color{incolor}31}]:} \PY{k+kn}{import} \PY{n+nn}{gmpy}
         \PY{n}{p} \PY{o}{=} \PY{n}{gmpy}\PY{o}{.}\PY{n}{mpz}\PY{p}{(}\PY{l+m+mi}{63689}\PY{p}{)}
         \PY{n}{g} \PY{o}{=} \PY{n}{gmpy}\PY{o}{.}\PY{n}{mpz}\PY{p}{(}\PY{l+m+mi}{14569}\PY{p}{)}
         \PY{n}{k} \PY{o}{=} \PY{n}{gmpy}\PY{o}{.}\PY{n}{mpz}\PY{p}{(}\PY{l+m+mi}{11636}\PY{p}{)}
\end{Verbatim}

    \textbf{Requirements:}

\textbf{1.} Number \texttt{p} needs to be prime. gmpy has a function
\texttt{is\_prime} which returns 2 when its argument is prime. Since
\texttt{p} is prime, its greatest common divisor with \texttt{g} is
clearly 1.

    \begin{Verbatim}[commandchars=\\\{\}]
{\color{incolor}In [{\color{incolor}15}]:} \PY{l+m+mi}{2} \PY{o}{==} \PY{n}{gmpy}\PY{o}{.}\PY{n}{is\PYZus{}prime}\PY{p}{(}\PY{n}{p}\PY{p}{)} \PY{o}{\PYZam{}}\PY{o}{\PYZam{}} \PY{l+m+mi}{1} \PY{o}{==} \PY{n}{gmpy}\PY{o}{.}\PY{n}{gcd}\PY{p}{(}\PY{n}{p}\PY{p}{,} \PY{n}{g}\PY{p}{)}
\end{Verbatim}

            \begin{Verbatim}[commandchars=\\\{\}]
{\color{outcolor}Out[{\color{outcolor}15}]:} True
\end{Verbatim}
        
    \textbf{2.} Number \texttt{g} needs to be a generator. Thus it must be a
member of the cyclic group $(Z_p)^*$ which in this case contains all the
numbers from 2 smaller than \texttt{p} because \texttt{p} is prime.
Moreover, it must be able to generate all of the numbers in this cyclic
group when raised to the power of each of the elements in $(Z_p)^*$ up
to $p - 2$. Therefore, the set generated by \texttt{g} and the set of
numbers from 1 to \texttt{p} - 1 should be equal:
$\{g^1, g^2, ..., g^{p-2}\} = \{1, 2, ..., p - 1\}$.

    \begin{Verbatim}[commandchars=\\\{\}]
{\color{incolor}In [{\color{incolor}36}]:} \PY{n}{g} \PY{o+ow}{in} \PY{n+nb}{range}\PY{p}{(}\PY{l+m+mi}{2}\PY{p}{,} \PY{n}{p}\PY{p}{)} \PY{o+ow}{and} \PY{n+nb}{set}\PY{p}{(}\PY{n+nb}{pow}\PY{p}{(}\PY{n}{g}\PY{p}{,} \PY{n}{i}\PY{p}{,} \PY{n}{p}\PY{p}{)} \PY{k}{for} \PY{n}{i} \PY{o+ow}{in} \PY{n+nb}{range}\PY{p}{(}\PY{l+m+mi}{1}\PY{p}{,} \PY{n}{p} \PY{o}{\PYZhy{}} \PY{l+m+mi}{1}\PY{p}{)}\PY{p}{)} \PY{o}{==} \PY{n+nb}{set}\PY{p}{(}\PY{n+nb}{range}\PY{p}{(}\PY{l+m+mi}{2}\PY{p}{,} \PY{n}{p}\PY{p}{)}\PY{p}{)}
\end{Verbatim}

            \begin{Verbatim}[commandchars=\\\{\}]
{\color{outcolor}Out[{\color{outcolor}36}]:} True
\end{Verbatim}
        
    \textbf{3.} Furthermore, the private key \texttt{k} needs to be from the
set $\{1, ..., p - 2\}$.

    \begin{Verbatim}[commandchars=\\\{\}]
{\color{incolor}In [{\color{incolor}33}]:} \PY{n}{k} \PY{o}{\PYZlt{}} \PY{n}{p} \PY{o}{\PYZhy{}} \PY{l+m+mi}{1}
\end{Verbatim}

            \begin{Verbatim}[commandchars=\\\{\}]
{\color{outcolor}Out[{\color{outcolor}33}]:} True
\end{Verbatim}
        
    \subsubsection{b)}\label{b}

To find the decryption of the ciphertext we use \texttt{gmpy} again.
Firstly, we are given the results of encryption: $c_1$ and $c_2$.
Secondly, to decrypt we use the decryption formula from the ElGamal
algorithm:

$D_{EG} = c_1^{-k} \cdot c_2 \pmod{p}$:

    \begin{Verbatim}[commandchars=\\\{\}]
{\color{incolor}In [{\color{incolor}34}]:} \PY{n}{c1}\PY{p}{,} \PY{n}{c2} \PY{o}{=} \PY{l+m+mi}{7265}\PY{p}{,} \PY{l+m+mi}{44824}
         \PY{n+nb}{pow}\PY{p}{(}\PY{n}{c1}\PY{p}{,} \PY{n}{p} \PY{o}{\PYZhy{}} \PY{n}{k} \PY{o}{\PYZhy{}} \PY{l+m+mi}{1}\PY{p}{)} \PY{o}{*} \PY{n}{c2} \PY{o}{\PYZpc{}} \PY{n}{p}
\end{Verbatim}

            \begin{Verbatim}[commandchars=\\\{\}]
{\color{outcolor}Out[{\color{outcolor}34}]:} mpz(2014)
\end{Verbatim}
        
    \subsection{Exercise 2}\label{exercise-2}

\subsubsection{a)}\label{a}

We could use a key $k$ of length at least as long as the message. Then
for each block $m_i$ of the message we would use a corresponding block
$k_i$ of the same size as $m_i$ from key $k$.

\subsubsection{b)}\label{b}

\begin{itemize}
\item
  Raw RSA does not satisfy this property. It is because RSA is
  deterministic, hence whenever we encrypt a message \texttt{m} with the
  same public key \texttt{pk} we always get the same ciphertext
  \texttt{c}. This is what the attacker could check for both messages
  and confirm which one encrypts to the ciphertext.
\item
  ElGamal does satisfy the property. It uses a random number \texttt{r}
  from each of the participants, hence the same message does not always
  encrypt to the same ciphertext. Furthermore, since we are using
  randomness, a message \texttt{m} can be encrypted to ciphertext
  \texttt{c} if we select the right random number \texttt{r}.
\end{itemize}

\subsubsection{c)}\label{c}

To show that ElGamal is malleable, we take $e$, $c_1$ which were
generated in encryption phase of ElGamal for message $m_1$. Now we can
apply the function $f(x) = 2x$ to $c_1$ so that we get:

$c_2 = f(c_1) = 2 \cdot m \cdot (g^d)^r \pmod{p}$.

Now, we send the tuple $e$, $c_2$ and when that is decrypted the message
$m_2$ is received instead of $m_1$:

\[ e^{-d} \cdot c_2 \pmod{p} = (g^r)^{-d} \cdot 2 \cdot m \cdot (g^d)^r \pmod{p} = 2 \cdot m \pmod{p}\]

\subsubsection{d)}\label{d}

Ciphertexts are rejected when the $k_1$ least significant bits of the
decrypted message $m$ are not equal to $0^{k_1}$, where $k_1$ is the
number of zeros that were padded to the end of message $m$ in the
encryption stage.

    \subsection{Exercise 3}\label{exercise-3}

\subsubsection{a)}\label{a}

We use a variant of Lowe's attack where \texttt{M} is the malicious user
who manages to convince \texttt{B} that he is \texttt{A}. This attack is
only possible if \texttt{A} tries to participate in this protocol with
\texttt{M}:

\begin{enumerate}
\def\labelenumi{\arabic{enumi}.}
\itemsep1pt\parskip0pt\parsep0pt
\item
  Let's say that \texttt{A} generates key \texttt{K} for communication
  with \texttt{M}.
\item
  After that \texttt{M} initiates an exchange with \texttt{B} with
  \texttt{A}'s identity and re-using the key \texttt{K} generated by
  \texttt{A} which \texttt{M} can read because it was encrypted by
  \texttt{M}'s public key.
\item
  Now, \texttt{B} generates a nonce \texttt{N} and sends it encrypted
  with key \texttt{K} to \texttt{M}.
\item
  \texttt{M} passes \texttt{N} encrypted with \texttt{K} to \texttt{A}
  who responds with his certificate and signed nonce \texttt{N} all
  encrypted with \texttt{K}.
\item
  Now, \texttt{M} only needs to pass this along to \texttt{B} for
  \texttt{B} to think that it is authenticating with \texttt{A} instead
  of \texttt{M}.
\item
  After the last step, all communication between \texttt{M} and
  \texttt{B} is encrypted with key \texttt{K} with \texttt{B} thinking
  that it is talking to \texttt{A} and \texttt{M} being able to decrypt
  the messages.
\end{enumerate}

The attack is described on the diagram below:

    \begin{Verbatim}[commandchars=\\\{\}]
{\color{incolor}In [{\color{incolor}3}]:} \PY{n}{Image}\PY{p}{(}\PY{l+s}{\PYZsq{}}\PY{l+s}{lowe\PYZus{}attack.png}\PY{l+s}{\PYZsq{}}\PY{p}{)}
\end{Verbatim}
\texttt{\color{outcolor}Out[{\color{outcolor}3}]:}
    
    \begin{center}
    \adjustimage{max size={0.9\linewidth}{0.9\paperheight}}{Computer Security Assignment 1_files/Computer Security Assignment 1_12_0.png}
    \end{center}
    { \hspace*{\fill} \\}
    

    \subsubsection{b)}\label{b}

We could fix this issue by adding in \texttt{B}'s (the receiver's)
identity in the last step of the protocol so that \texttt{A} sends
\texttt{B} $senc(K, Cert_A \vert \vert sign(pvk(A), <N, B>))$. Thus
\texttt{M} would not be able to send this message along to \texttt{B}
pretending to be \texttt{A} because it would be his identity included in
the signature as a receiver and not \texttt{B}'s identity.

Furthermore, \texttt{M} would not be able to change it to \texttt{B}'s
identity since the signature is signed by the private key of \texttt{A}.

    \pagebreak

    \subsection{Exercise 4}\label{exercise-4}

\subsubsection{a)}\label{a}

\textbf{Properties}

\begin{itemize}
\itemsep1pt\parskip0pt\parsep0pt
\item
  Both participants get an equal say in what the resulting number should
  be. Thus none of them is able to force it (as required).

  \begin{itemize}
  \itemsep1pt\parskip0pt\parsep0pt
  \item
    My protocol achieves this by getting a random number from both
    participants and XOR-ing it together to form a new number. There is
    no way for any of the participants to force some number because they
    will not know what number the other person has chosen until they
    commit to a number themselves.
  \end{itemize}
\item
  The participants should not be able to change the number they
  generated based on the other participant's number.

  \begin{itemize}
  \itemsep1pt\parskip0pt\parsep0pt
  \item
    My protocol achieves this by forcing the players to commit to the
    hashes of their numbers. Thus the participants are not able to
    change their numbers after they reveal their hashes because the
    number would not hash to the same hash (we use a collision-resistant
    function for hashing). Furthermore, they are not able to deduce what
    number the other participant generated before they commit to their
    number because the hashing function is one-way.
  \end{itemize}
\end{itemize}

\textbf{Requirements}

\begin{itemize}
\itemsep1pt\parskip0pt\parsep0pt
\item
  The participants have agreed on a CRF hash function $H(x)$ beforehand
  which they can use to verify that the other participant's hash
  corresponds to their number.
\item
  The participants are able to hash the number they generated with
  $H(x)$ and are able to verify other players' hashes.
\end{itemize}

\textbf{Assumptions}

\begin{itemize}
\itemsep1pt\parskip0pt\parsep0pt
\item
  The randomly generated number does not need to be secret so the
  players do not mind an attacker eavesdropping.
\item
  The phones are tamper-free and the Man in the Middle attack is not
  possible. Otherwise, the attacker could pretend to be the other
  participant to both of the participants and generate a random number
  that he wants.
\end{itemize}

\textbf{Protocol Steps}

\begin{enumerate}
\def\labelenumi{\arabic{enumi}.}
\itemsep1pt\parskip0pt\parsep0pt
\item
  Both participants choose a random number.
\item
  They hash this number with a CRF hash function $H(x)$.
\item
  Both participants reveal these hashes and commit to them.
\item
  Afterwards, they reveal their random numbers.
\item
  Both of the participants can check that the random number of the other
  participant hashes to the commited value using the function $H(x)$.
\item
  The random numbers are XOR-ed together to form a new randomly
  generated number.
\end{enumerate}

    \subsubsection{b)}\label{b}

\textbf{Properties}

\begin{itemize}
\itemsep1pt\parskip0pt\parsep0pt
\item
  The participants should not be able to change the hand-signs they want
  to play based on the other participant's hand-sign.

  \begin{itemize}
  \itemsep1pt\parskip0pt\parsep0pt
  \item
    My protocol achieves this by forcing the players to commit to the
    hashes of their hand-signs padded with a random number. Thus the
    participants are not able to change their hand-signs or random
    numbers after they reveal their hashes because they would not hash
    to the same hash (we use a collision-resistant function for
    hashing). Furthermore, they are not able to deduce what hand-sign
    the other participant played before they commit to their hand-sign
    because the hashing function is one-way and the other participant
    used an unknown random number to pad the value of their hand-sign
    before hashing.
  \end{itemize}
\end{itemize}

\textbf{Requirements}

\begin{itemize}
\itemsep1pt\parskip0pt\parsep0pt
\item
  The participants have agreed on a CRF hash function $H(x)$ beforehand
  which they can use to verify that the other participant's hash
  corresponds to their number.
\item
  The participants have agreed on how they are going to perform the
  padding.
\item
  The participants are able to hash the number they generated with
  $H(x)$ and are able to verify other players' hashes.
\end{itemize}

\textbf{Assumptions}

\begin{itemize}
\itemsep1pt\parskip0pt\parsep0pt
\item
  The game is played on a secure channel so that an attacker cannot
  change players' hand-signs.
\item
  The hand-signs do not need to be secret so that if an attacker is
  eavesdropping it has no unwanted consequences.
\end{itemize}

\textbf{Protocol Steps}

\begin{enumerate}
\def\labelenumi{\arabic{enumi}.}
\itemsep1pt\parskip0pt\parsep0pt
\item
  Each participant chooses what hand-sign they are going to play.
\item
  They add some padding to the value of the hand-sign using a randomly
  generated number.
\item
  They hash the padded value with a CRF $H(x)$ and reveal the hash -
  commiting to it.
\item
  Afterwards, they reveal their randomly generated number and their
  hand-sign - commiting to it.
\item
  After that, the players are able to verify each other's hand-sign by
  padding the hand-sign with the randomly generated number and hashing
  it using $H(x)$.
\item
  After that, all the hand-signs are verified and we can play the round.
\end{enumerate}


    % Add a bibliography block to the postdoc
    
    
    
    \end{document}
